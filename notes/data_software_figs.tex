\section {Data Files, Software, and Figures}

\subsection{Data Files}
\begin{itemize}
    \item \textbf{Binary}:
    \begin{itemize}
        \item Advantages: compact file size.
        \item Disadvantages: need encoders (otherwise difficult to read).
        \item Microsoft 365 files (.xlsx, .docx, etc.), which are proprietary but have been open-sourced for interoperability with, for example, Apple and Google's suites.
        \item \textbf{.fits}: Well-known, well-defined, and used by the astronomy community.
        \item \textbf{.cdf}: Common Data Format, standardized encoding.
        \item \textbf{.sav}: IDL's proprietary format.
    \end{itemize}
    \item \textbf{ASCII}:
    \begin{itemize}
        \item Advantages: built-in encoding table and, thus, easy to read.
        \item Disadvantages: large file size.
        \item .txt, .csv, etc.
    \end{itemize}
\end{itemize}

\subsection{Software}
\begin{itemize}
    \item A wide array of scientific software exists such as MATLAB and IDL.
    \item However, these tend to require expensive licenses.
    \item In contrast, Python is free and open-source.
    \item However, Python comes with no warranty, meaning that if something goes wrong, the user is responsible.
\end{itemize}

\subsection{Figures}
A professional-quality figure
\begin{itemize}
    \item is grayscale, relying on various marker sizes and shapes (e.g. circle, diamond, etc.) as well as line styles (solid, dashed, etc.) to differentiate different series.
    \item contains a title, axes labels with units, and a legend as necessary.
    \item uses space on the plot effectively.
    \item has proper axis scaling.
    \item is high resolution (usually 600 DPI or more).
\end{itemize}