\section {Introduction to Research}

Some information is specific to the New Jersey Institute of Technology (NJIT).

\begin{itemize}
    \item A typical external federal grant is \$130,000 for 3 years.
    \item Professors must bring in grants to be tenured.
    \item Professors also have a "2 + 2" teaching requirement: 2 courses in the fall, and 2 courses in the spring.
    \item If a professor supports a research assistant (RA), covering their \textbf{stipend}, \textbf{tuition}, and \textbf{fringe}, they go down to 1 + 2.
    \item Supporting another RA reduces this to 1 + 1.
    \item In contrast, teaching assistants (TAs) are paid for by the university, not the professor.
    \item A researcher or RA typically cannot have another job if they are already being funded by a grant for maximal dedication.
    \item Usually, grant money goes to the university, not the professor directly.
    \item PhDs in the United States usually take 6 years because new knowledge must be created to attain it.
    \item A typical physics graduate student coursework:
    \begin{itemize}
        \item Fall: Electricity and Magnetism I, Classical Mechanics, Mathematical Methods
        \item Spring: Electricity and Magnetism II, Quantum Mechanics, Statistical Mechanics / Thermodynamics
    \end{itemize}
    \item Qualifying exams in the summer (June for NJIT) determine whether a student remains in the graduate program.
    \item Qualifying exams are difficult (typically 50\% pass rate for NJIT).
    \item There is one last redemption exam in January for those who fail.
    \item In the 1980s, program officers would evaluate grant proposals, but from 1990s onward, panels of qualified individuals perform this task.
\end{itemize}

\subsection{Federal Funding Agencies}

\begin{itemize}
    \item Top agencies that fund grants:
    \begin{itemize}
        \item National Science Foundation (NSF)
        \item Department of Energy (DOE)
        \item National Aeronautics and Space Administration (NASA)
        \item National Institutes of Health (NIH)
        \item Department of Defense (DoD)
    \end{itemize}
    \item All except NSF require \textbf{closure}, which means meaningful results must be found, and the research question must be answered.
    \item Some agencies require an American citizenship for all researchers.
\end{itemize}

\subsection{Grant Format}
\begin{itemize}
    \item A small part of the grant ($\sim 10\%$) is:
    \begin{itemize}
        \item Project summary (1 page)
        \item Project description (15+ pages)
        \item Export control, etc.
    \end{itemize}
    \item But the majority of the grant ($\sim 90\%$) is paperwork
\end{itemize}

\subsection{Fringe and Overhead}
\begin{itemize}
    \item \textbf{Fringe} is the cost of employing someone, which includes associated costs such as insurance, retirement, etc.
    \item \textbf{Overhead} is the money used to support the research enterprise of an institution, which includes the cost of utilities, supplies, subscriptions, etc.
    \item The fringe at NJIT went from 43\% to 86\% in 5 years.
    \item The cost of research has skyrocketed due to rising insurance premiums and unionization.
\end{itemize}